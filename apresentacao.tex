\documentclass{beamer}
    \usetheme{Berlin}
    \usecolortheme{crane}
    \usepackage[brazilian]{babel}
    \usepackage[utf8]{inputenc}
    \usepackage[T1]{fontenc}
    \usepackage{graphicx}
    \title{Azure CosmosDB}
    \author[Athos Castro Moreno]{Athos Castro Moreno}
    \institute{Universidade Tecnológica Federal do Paraná}
    \titlegraphic{\includegraphics[scale=0.15]{utfpr-logo.png}}
    \date{\today}
\begin{document}
    \begin{frame}
        \titlepage
    \end{frame}
    \begin{frame}
        \frametitle{Sumário}
        \tableofcontents
    \end{frame}
    \section{Introdução}
    \begin{frame}
        \frametitle{Introdução}
        \begin{itemize}
             \item Banco de dados NoSQL;
             \begin{itemize}
                 \item O que é;
                 \item Aplicação e utilização;
             \end{itemize}
             \item Serviços autogerenciaveis;
             \item Azure CosmosDB.
        \end{itemize}
    \end{frame}
    \section{CosmosDB}
    \begin{frame}
        \frametitle{CosmosDB}
        \begin{itemize}
            \item Serviço mantido pela Microsoft;
            \item Criado em 2010 como um projeto interno chamado Florence;
            \item Uso em aplicações de larga escala da Microsoft;
            \item Em 2015 foi disponibilizado no Azure com o nome DocumentDB;
            \item Em 2017 foi renomeado CosmosDB.
        \end{itemize}
    \end{frame}
    \begin{frame}
        \frametitle{Vantagens}
        \begin{itemize}
            \item Não é necessária manutenção;
            \item Funciona na nuvem da Azure;
            \item Georeplicação via portal da Azure;
            \item Possibilidade de utilizar diferentes implementações NoSQL;
        \end{itemize}
    \end{frame}
\end{document}