\documentclass[12pt]{article}
\usepackage{sbc-template}
\usepackage{graphicx,url}
\usepackage{hyperref}
\usepackage[alf]{abntex2cite}
\usepackage[brazil]{babel}   
\usepackage[utf8]{inputenc} 
     
\sloppy

\title{Azure CosmosDB: Estudo e aplicação}

\author{Athos Castro Moreno}

\address{Departamento de Computação - Universidade Tecnológica Federal do Paraná
  (UTFPR)\\
  Avenida Alberto Carazzai, 1640 -- 86300000 -- Cornélio Procópio -- PR -- Brasil
  \email{athos@alunos.utfpr.edu.br}
}

\begin{document} 

\maketitle

\begin{resumo} 
\end{resumo}

\section{Introdução}

\section{Problematização} 

\subsection{NoSQL}
Na industria de tecnologia de informação, é comum a utilização dos bancos de dados relacionais que usufruem da linguagem SQL para as operações. Porém, com a crescente demanda
de maleabilidade na modelagem de um banco de dados em empresas com requisitos que não estão com uma estrutura de dados bem definida ou em constante mudança, foi necessária a 
utilização de soluções além de bancos de dados relacionais.

O conceito de bancos de dados não relacionais existe desde 1960 e reganhou popularidade em 2007, principalmente com o anúncio do Amazon DynamoDB, serviço NoSQL na Amazon Web Service.
A primeira implementação NoSQL foi criada por Carlo Stozzi, aonde um banco de dados \textit{open source} em shell não utilizava nenhuma interface SQL para ser manipulado. 

Cada implementação NoSQL possuí uma abordagem diferente, tais como \cite{Lith2010}:

\begin{itemize}
	\item \textbf{\textit{Key-value}:} Armazena um par de chaves e valores, funcionando com um \textit{array} associativo. Pode armazenar dados estruturados ou dados sem estrutura.
	\item \textbf{Orientado a documentos:} Uma subclasse do \textit{key-value}, tendo diferença no processamento dos dados através da estrutura interna do documento. Podem ser utilizados 
	os formatos de arquivos XML, YAML, JSON, BSON, entre outras opções.
	\item \textbf{Orientado a colunas:} Nessa abordagem as informações são armazenadas em colunas. É otimizado para recuperação rápida de dados, uso em aplicativos 
	analíticos e reduz requerimentos de E/S de disco e quantidade de dados necessário a serem carregados.
\end{itemize}

O uso de um banco de dados NoSQL é recomendado para ambientes dinâmicos, que precisem ser escaláveis entre várias máquinas de maneira mais rápida, com uma estrutura de dados que não está bem 
definida ou em constante mudança. É importante ressaltar que o NoSQL não tem como foco substituir todos os bancos relacionais existentes e sim fornecer uma alternativa para os que 
precisam de mais velocidade e estar menos presos a uma modelagem \cite{Oliveira2011} \cite{Lith2010} \cite{Leavitt2010}. 

\subsection{Azure CosmosDB}
Além das vantagens do NoSQL, o CosmosDB fornece outras funcionalidades \cite{Paz2018}:

\begin{itemize}
	\item Georeplicação via portal nos \textit{datacenters} da Azure;
	\item Baixa latência de leitura/escrita;
	\item Serviço autogerenciado;
	\item Suporta múltiplos \textit{data models};
	\item Escalabilidade de custo;
	\item \textit{Failover} automático, se configurado.
\end{itemize}

É importante lembrar que o CosmosDB, apesar de utilizar diversas tecnologias \textit{open source}, é um serviço proprietário da Microsoft e seu uso é limitado dentro dos 
servidores da Azure.

\bibliography{bibliografia}

\end{document}
