\documentclass[12pt]{article}
\usepackage{sbc-template}
\usepackage{graphicx,url}
\usepackage{hyperref}
\usepackage[alf]{abntex2cite}
\usepackage[brazil]{babel}   
\usepackage[utf8]{inputenc} 
     
\sloppy

\title{Azure CosmosDB: Estudo e aplicação}

\author{Athos Castro Moreno}

\address{Departamento de Computação - Universidade Tecnológica Federal do Paraná
  (UTFPR)\\
  Avenida Alberto Carazzai, 1640 -- 86300000 -- Cornélio Procópio -- PR -- Brasil
  \email{athos@alunos.utfpr.edu.br}
}

\begin{document} 

\maketitle
     
\begin{resumo} 
\end{resumo}

\section{Introdução}

\section{Problematização} 

O Azure CosmosDB é um banco de dados NoSQL com suporte a diversas APIs, tais como as do MongoDB, um popular banco de dados NoSQL. A funcionalidade de um banco de dados NoSQL é diferente dos RDBMS (Rational Database Management System) existentes e utilizados no mercado (MySQL, PostgreSQL, Microsoft SQL Server, etc), tendo que plataformas NoSQL tem um foco diferente e se propõe a resolver alguns problemas encontrados no uso de RDBMS, de acordo com \cite{Membrey2011} e \cite{Oliveira2011}:
\begin{itemize}
	\item Fácil escalabilidade horizontal;
	\item Perfomance;
	\item \textit{Sharding} (diferentes partes de um \textit{dataset} espalhadas por multíplos servidores);
	\item Uso apropriado para dados não estruturados.
\end{itemize}

\bibliography{bibliografia}

\end{document}
