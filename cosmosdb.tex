\documentclass[12pt]{article}
\usepackage{sbc-template}
\usepackage{graphicx,url}
\usepackage{hyperref}
\usepackage[alf]{abntex2cite}
\usepackage[brazil]{babel}   
\usepackage[utf8]{inputenc} 
     
\sloppy

\title{Azure CosmosDB: Estudo e aplicação}

\author{Athos Castro Moreno}

\address{Departamento de Computação - Universidade Tecnológica Federal do Paraná
  (UTFPR)\\
  Avenida Alberto Carazzai, 1640 -- 86300000 -- Cornélio Procópio -- PR -- Brasil
  \email{athos@alunos.utfpr.edu.br}
}

\begin{document} 

\maketitle

\begin{resumo} 
\end{resumo}

\section{Introdução}

\section{Problematização} 

\subsection{NoSQL}
Na industria de tecnologia de informação, é comum a utilização dos bancos de dados relacionais que usufruem da linguagem SQL para as operações. Porém, com a crescente demanda
de maleabilidade na modelagem de um banco de dados em empresas com requisitos que não estão com uma estrutura de dados bem definida ou em constante mudança, foi necessária a 
utilização de soluções além de bancos de dados relacionais.

O Azure CosmosDB é um banco de dados NoSQL com suporte a diversas APIs, tais como as do MongoDB, um popular banco de dados NoSQL. A funcionalidade de um banco de dados 
NoSQL é diferente dos bancos relacionais existentes e utilizados no mercado (MySQL, PostgreSQL, Microsoft SQL Server, etc), tendo que plataformas 
NoSQL tem um foco diferente e se propõe a resolver alguns problemas encontrados no uso de bancos relacionais, de acordo com \cite{Membrey2011} e \cite{Oliveira2011}:
\begin{itemize}
	\item Fácil escalabilidade horizontal;
	\item Performance;
	\item \textit{Sharding} (diferentes partes de um \textit{dataset} espalhadas por múltiplos servidores);
	\item Uso apropriado para dados não estruturados.
\end{itemize}

O uso de um banco de dados NoSQL é recomendado para ambientes dinâmicos, que precisem ser escaláveis entre várias máquinas de maneira mais rápida, com uma estrutura de dados que não está bem 
definida ou em constante mudança. É importante ressaltar que o NoSQL não tem como foco substituir todos os bancos relacionais existentes e sim fornecer uma alternativa para os que 
precisam de mais velocidade e estar menos presos a uma modelagem. 

\subsection{Azure CosmosDB}
Além das vantagens do NoSQL, o CosmosDB fornece outras funcionalidades:

\begin{itemize}
	\item Georeplicação via portal nos \textit{datacenters} da Azure;
	\item Baixa latência de leitura/escrita;
	\item Serviço autogerenciado;
	\item Suporta múltiplos \textit{data models};
	\item Escalabilidade de custo;
	\item \textit{Failover} automático, se configurado.
\end{itemize}

É importante lembrar que o CosmosDB, apesar de utilizar diversas tecnologias \textit{open source}, é um serviço proprietário da Microsoft e seu uso é limitado dentro dos 
servidores da Azure.

\bibliography{bibliografia}

\end{document}
